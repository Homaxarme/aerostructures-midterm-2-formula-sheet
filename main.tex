\documentclass{article}
\usepackage{mathtools}
\usepackage[margin=1in]{geometry}
\title{Aerostructures Exam II Formula Sheet}
\author{Gabriel Duany Izaguirre}
\date{}

\begin{document}
\maketitle
\section{Euler Column Bending}
\begin{equation}
P_{cr} = \frac{\pi ^ 2 EI}{KL^2}
\end{equation}
where
\begin{itemize}
    \item $P_{cr},$ Euler Critical Load,
    \item $E,$ Modulus of Elasticity,
    \item $K,$ Effective Length Factor,
    \item $L,$ Column Length.
\end{itemize}
\section{Crippling Stress}
\subsection{Gerard Method}
The Gerard method has two equations, those that apply to Channel and Zee sections, and those that apply to all other sections.
For all other sections, you must determine the number of flanges and cuts, the $g$ value.
Equation~\ref{eqn:gerard-zee-channel} is used for Channel and Zee sections while Equation~\ref{eqn:gerard-others}
\begin{equation}\label{eqn:gerard-zee-channel}
    F_{cc} = B\left(\frac{1}{\frac{A}{t^2}}\right)^m\sqrt[3]{E^mF_{cy}^{(3-m)}}
\end{equation}
where $B = 4.05$ and $m=0.82$.

\begin{equation}\label{eqn:gerard-others}
   F_{cc} = B\left(\frac{1}{\frac{A}{gt^2}}\right)^m\sqrt{E^mF_{cy}^{(2-m)}}
\end{equation}
where $B = 0.58$ and $ m = 0.8$

\subsection{Bruhn Method}

\end{document}