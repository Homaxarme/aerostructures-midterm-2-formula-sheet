\documentclass{article}
\usepackage{mathtools}
\usepackage[utf8x]{inputenc}
\usepackage[margin=1in]{geometry}
\title{Aerostructures  Exam Zwei Formula Sheet}
\author{}
\date{}
\begin{document}
\maketitle
\section{Beam Columns}
Grandaddy Equation of Beam Buckling:
\begin{equation}\label{eqn:grandaddy}
P_{cr} = \frac{\pi ^ 2 EI}{KL^2}
\end{equation}
where
\begin{itemize}
    \item $P_{cr},$ Euler Critical Load,
    \item $E,$ Modulus of Elasticity,
    \item $K,$ Effective Length Factor,
    \item $L,$ Column Length.
\end{itemize}
\subsection{Approximation Method for Beam Moment}
\begin{equation}
    M = \cfrac{M_o}{1 - \cfrac{P}{P_{cr}}}
\end{equation}


\section{Crippling Stress}
\subsection{Gerard Method}
The Gerard method has two equations, those that apply to Channel and Zee sections, and those that apply to all other sections.
For all other sections, you must determine the number of flanges and cuts, the $g$ value.
Equation~\ref{eqn:gerard-zee-channel} is used for Channel and Zee sections while Equation~\ref{eqn:gerard-others} is used for all other sections.
There is a table that gives the $g$ values for certain shapes.
\begin{equation}\label{eqn:gerard-zee-channel}
    F_{cc} = B\left(\frac{1}{\frac{A}{t^2}}\right)^m\sqrt[3]{E^mF_{cy}^{(3-m)}}
\end{equation}
where $B = 4.05$ and $m=0.82$.

\begin{equation}\label{eqn:gerard-others}
   F_{cc} = B\left(\frac{1}{\frac{A}{gt^2}}\right)^m\sqrt{E^mF_{cy}^{(2-m)}}
\end{equation}
where $B = 0.58$ and $ m = 0.8$

\subsection{Bruhn Method}
Divide the shape into different sections and calculate the individual stresses $\sigma$ using Equation~\ref{eqn:bruhn}.
Use centerline dimensions.
If you get a value above the yield strength of the material $\sigma_y$, the allowable stress is then the yield stress.
$K_c$ is the buckling coefficient, which will have to be obtained from a graph.
Once you have calculated the individual stresses, calculate the total force and divide by the total area to get the average stress (Equation~\ref{eqn:average-stress})
\begin{equation}\label{eqn:bruhn}
    \sigma = \frac{K_c \pi ^ 2 E}{12(1 - \nu_e ^ 2)}\left(\frac{t}{b}\right) ^ 2
\end{equation}
\begin{equation}\label{eqn:average-stress}
    \sigma_{avg} = \frac{\sum\sigma_i A_i}{\sum A_i}
\end{equation}
For aluminum sections, we can use the following equations to calculate $\sigma$, one for free at one end and simply supported at the other (Equation~\ref{eqn:simply-free}), and another for simply supported at both ends (Equation~\ref{eqn:simply-simply}).
\begin{equation}\label{eqn:simply-free}
   \sigma = 0.452E\left(\frac{t}{b}\right) ^ 2
\end{equation}
\begin{equation}\label{eqn:simply-simply}
    \sigma = 3.62E\left(\frac{t}{b}\right) ^ 2
\end{equation}
\subsubsection{Correction For Plasticity}
Once $\sigma_{avg}$ is calculated, divide it by $F_{0.7}$ and use the Plasticity Correction graph from Bruhn to determine the corrected Crippling Stress.
Once the corrected crippling stress is deterimned, total allowable load can be calculated by multiplying the allowable stress by the area.
\section{Flexural Instability}
Crippling will cause a beam to fail before Grandaddy Euler Equation~\ref{eqn:grandaddy} predicts.
Therefore, to take into account crippling, use Equation~\ref{eqn:flexural-instability}.
Remember that $\rho$ is $\sqrt{\frac{I}{A}}$, or the radius of gyration.
The c value depends on the type of rivets that are being used; default to 1.
\begin{equation}\label{eqn:flexural-instability}
    F_c = F_{cc}\left[1 - \frac{F_{cc} \left(\frac{l}{\rho \sqrt{c}}\right) ^ 2}{4\pi ^2 E}\right]
\end{equation}
\section{Inter-Rivet Buckling}
The skin between the rivets will buckle and to prevent that, you should space the rivets closer together.
If you know the flexural instability of the object, $F_c$, then you can set that to the inter-rivet buckling stress, because you are screwed if you exceed that value.
The stress that can be taken before inter-rivet buckling occurs, then you can use Equation~\ref{eqn:inter-rivet-stress}
\begin{equation}\label{eqn:inter-rivet-stress}
    F_{ir} = 0.9cE\left(\frac{t}{s}\right) ^ 2
\end{equation}
To calculate inter-rivet pitch, use Equation~\ref{eqn:inter-rivet-pitch}
Using graphs, we can calculate the inter-rivet buckling stress by looking at $\frac{s}{t}$ and finding th stress that matches.
That graph assumes that $c=4$, but if it is not 4, then you need to use a correction factor $\sqrt{\frac{c}{5}}$.
\begin{equation}\label{eqn:inter-rivet-pitch}
   s = t\sqrt{\frac{0.9cE}{F_{ir}}} 
\end{equation}
\section{Torsion in thin-walled closed sections}
For any section, the angle of twist is given by the following equation:
\begin{equation}
\theta = \frac{TL}{JG}
\end{equation}
Where $J$ is the torsional constant, the polar moment of inertia for a circular cross-section.
For noncircular cross-sections, then $J$ can be calculated as follows:

\begin{equation}
    J = \frac{4A^2}{\sum_{i=1}^n\frac{b_i}{t_i}}
\end{equation}

The shear flow and torsional shear stress are:
\begin{equation}
    q = \frac{T}{2A}
\end{equation}
\begin{equation}
    f_{ts} = \frac{T}{2At}
\end{equation}
\section{Tapered Beams}
You can make a table for this that lists: $x$, $h$, $\frac{h_0}{h}$, $V_w = V(\frac{h_0}{h})$, $q =  \frac{V_w}{h}$.
To get the forces taken by the caps, you realize that the force acting vertically on the web is going to create a moment that will have to be counteracted.
This force is horizontal and is labeled as $P$.
Do a moment balance to determine the magnitude of B.
You can then use trigonometry to determine how much force the caps are taking.
There are also two equations given by the book:
\begin{equation}
    V_c = P(\tan\alpha_1 + \tan\alpha_2)
\end{equation}
\begin{equation}
    V_c = P(-\tan\alpha_1 + \tan\alpha_2)
\end{equation}

\end{document}
